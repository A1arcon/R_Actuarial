% Herramientas para beamer------------------------------------------------
\usepackage{ragged2e}
\usepackage{etoolbox}
\usepackage{lipsum}
\geometry{paperwidth=140mm,paperheight=125mm} % Tamaño de la hoja
% Para justificar el texto.
\apptocmd{\frame}{}{\justifying}{} 
\apptocmd{\column}{}{\justifying}{}
\renewcommand{\raggedright}{\leftskip=0pt \rightskip=0pt plus 0cm}
\addtobeamertemplate{block begin}{}{\justifying}  %new code
% Blocks de colores ~~~~~~~~~~~~~~~~~~~~~~~~~~~~~~~~~~~~~~~~~~~~~~~~~~~~~~~
% Amarillo
\newenvironment{yellow_block}[1]
{
\setbeamercolor{block title}{fg=black,bg=yellow!50!white}
\setbeamercolor{block body}{fg=black,bg=yellow!30!white}
\begin{block}{#1}
}
{
\end{block}
\setbeamercolor{block title}{fg=blocktitlefgsave,bg=blocktitlebgsave}
\setbeamercolor{block body}{fg=blockbodyfgsave,bg=blockbodybgsave}
}
% En general podemos definir así los bloques
\newenvironment{color_block}[2]
{
\setbeamercolor{block title}{fg=#1,bg=#1!50!white}
\setbeamercolor{block body}{fg=black, bg=#1!30!white}
\begin{block}{#2}
}
{
\end{block}
\setbeamercolor{block title}{fg=blocktitlefgsave,bg=blocktitlebgsave}
\setbeamercolor{block body}{fg=blockbodyfgsave,bg=blockbodybgsave}
}
% Proofs de colores ~~~~~~~~~~~~~~~~~~~~~~~~~~~~~~~~~~~~~~~~~~~~~~~~~~~~~~~
% Amarillo
\newenvironment{yellow_proof}[1]
{
\setbeamercolor{block title}{fg=black,bg=yellow!50!white}
\setbeamercolor{block body}{fg=black,bg=yellow!30!white}
\begin{proof}[#1]
}
{
\end{proof}
\setbeamercolor{block title}{fg=blocktitlefgsave,bg=blocktitlebgsave}
\setbeamercolor{block body}{fg=blockbodyfgsave,bg=blockbodybgsave}
}
% En general podemos definir así los bloques
\newenvironment{color_proof}[2]
{
\setbeamercolor{block title}{fg=#1,bg=#1!50!white}
\setbeamercolor{block body}{fg=black, bg=#1!30!white}
\begin{proof}[#2]
}
{
\end{proof}
\setbeamercolor{block title}{fg=blocktitlefgsave,bg=blocktitlebgsave}
\setbeamercolor{block body}{fg=blockbodyfgsave,bg=blockbodybgsave}
}
% ------------------------------------------------------------------------
% Para los hipervínculos -------------------------------------------------
\usepackage{hyperref}
\hypersetup{
    colorlinks=true,
    linkcolor=blue,
    filecolor=magenta,      
    urlcolor=cyan,
}
\urlstyle{same}
% ------------------------------------------------------------------------
% Paqueterías extra y funciones creadas.
\usepackage[spanish]{babel}
\newcommand{\derivada}[2]{\frac{\partial{#1}}{\partial{#2}}}

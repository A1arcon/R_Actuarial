% ------------------------------------------------------------------------
% Para los colores en el texto -------------------------------------------------
\usepackage{hyperref}
\hypersetup{
    colorlinks=true,
    linkcolor=black,
    filecolor=magenta,      
    urlcolor=cyan,
    citecolor=red,
	anchorcolor = blue
}
\urlstyle{same}
% ------------------------------------------------------------------------
% Paqueterías extra y funciones creadas.
\usepackage[spanish,es-nodecimaldot]{babel}
\usepackage{graphicx}
\usepackage{hyperref}
\newcommand{\derivada}[2]{\frac{\partial{#1}}{\partial{#2}}}
\usepackage{pdfpages} % Para poner pdf's en Markdown
\setcounter{tocdepth}{3} % Para el índice con profundidad 3
\usepackage[titles]{tocloft} %1. Para que no se vean los puntos en el índice
\renewcommand{\cftdot}{} %2. Para que no se vean los puntos en el índice
\usepackage{multicol}
\usepackage{amsmath,mathrsfs,amssymb,amsthm,commath,thmtools,enumerate}
\usepackage{tgtermes} %Letra monita.
\pagenumbering{gobble} %Para quitar el número de página.
\usepackage{caption}
\captionsetup[table]{name=Tabla}
\usepackage{fancyhdr}
\pagestyle{fancy}
\fancyhead[L]{Proyecto 2}
\fancyhead[C]{Estadística Espacial}
\fancyhead[R]{Alarcón González Edgar Gerardo}
\usepackage[spanish,es-tabla]{babel}
% ------------------------------------------------------------------------
% Para los colores en el texto -------------------------------------------------
\usepackage{hyperref}
\definecolor{colorref}{rgb}{0.65, 0.04, 0.37}
\hypersetup{
    colorlinks=true,
    linkcolor=colorref,
    filecolor=magenta,      
    urlcolor=cyan,
    citecolor=colorref,
	anchorcolor = blue
}
\usepackage{caption}
\captionsetup[table]{name=Tabla}
\captionsetup{labelfont={color=colorref}}
\def\figureautorefname{Figura}%
\renewcommand{\tableautorefname}{Tabla}
%\usepackage{cleveref}
\urlstyle{same}
% ------------------------------------------------------------------------
% Cambiar el color de \texttt{}
\usepackage{xcolor}
\definecolor{amber}{rgb}{1.0, 0.75, 0.0}
\definecolor{bronze}{rgb}{0.8, 0.5, 0.2}
\newcommand{\ctexttt}[2][bronze]{
\textcolor{#1}{\texttt{\textbf{#2}}}
}
% Ejemplo:
% \ctexttt[red]{Auto} % Pone "Auto" de color rojo
% \ctexttt{Auto} % Pone "Auto" de color bronze
% http://latexcolor.com/
% https://www.overleaf.com/learn/latex/Using_colours_in_LaTeX
% https://www.dickimaw-books.com/latex/novices/html/newcomopt.html
% ------------------------------------------------------------------------
% Poner gráficos y tablas lado a lado
\usepackage{floatrow}
\def\tablename{Tabla}
\newfloatcommand{btabbox}{table}
% ------------------------------------------------------------------------
% Paqueterías extra y funciones creadas.
\usepackage{graphicx}
\newcommand{\derivada}[2]{\frac{\partial{#1}}{\partial{#2}}}
\usepackage{pdfpages} % Para poner pdf's en Markdown
\setcounter{tocdepth}{2} % Para el índice con profundidad 3
\usepackage[titles]{tocloft} %1. Para que no se vean los puntos en el índice
\renewcommand{\cftdot}{} %2. Para que no se vean los puntos en el índice
\usepackage{multicol}
\usepackage{amsmath,mathrsfs,amssymb,amsthm,commath,thmtools,enumerate}
\usepackage{tgtermes} %Letra monita.
\pagenumbering{gobble} %Para quitar el número de página.
\usepackage{caption}
\usepackage{fancyhdr}
\pagestyle{fancy}
\fancyhead[L]{Aprendizaje estadístico automatizado}
\fancyhead[C]{Tarea-Examen 2}
\fancyhead[R]{Alarcón González Edgar Gerardo}